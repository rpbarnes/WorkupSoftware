\documentclass[10pt]{book}
\newif\ifshowall
\showalltrue
%\showallfalse
%\documentclass[twocolumn,nofootinbib]{revtex4}
\usepackage{cmap}
\usepackage[T1]{fontenc}
\usepackage{mynotebook}
%\usepackage[english,greek]{babel}
\synctex=1
%DIF PREAMBLE EXTENSION ADDED BY LATEXDIFF
%DIF UNDERLINE PREAMBLE %DIF PREAMBLE
\RequirePackage[normalem]{ulem} %DIF PREAMBLE
\RequirePackage{color}\definecolor{RED}{rgb}{1,0,0}\definecolor{BLUE}{rgb}{0,0,1} %DIF PREAMBLE
%\providecommand{\DIFadd}[1]{\begingroup\protect\color{blue}\uwave{#1}\endgroup} %DIF PREAMBLE
\providecommand{\DIFadd}[1]{#1} %DIF PREAMBLE
%\providecommand{\DIFdel}[1]{\begingroup\protect\color{red}\sout{#1}\endgroup}                      %DIF PREAMBLE
\providecommand{\DIFdel}[1]{#1}                      %DIF PREAMBLE
%DIF SAFE PREAMBLE %DIF PREAMBLE
\providecommand{\DIFaddbegin}{\color{blue}} %DIF PREAMBLE
\providecommand{\DIFaddend}{\color{black}} %DIF PREAMBLE
\providecommand{\DIFdelbegin}{} %DIF PREAMBLE
\providecommand{\DIFdelend}{} %DIF PREAMBLE
%DIF FLOATSAFE PREAMBLE %DIF PREAMBLE
\providecommand{\DIFaddFL}[1]{\DIFadd{#1}} %DIF PREAMBLE
\providecommand{\DIFdelFL}[1]{\DIFdel{#1}} %DIF PREAMBLE
\providecommand{\DIFaddbeginFL}{} %DIF PREAMBLE
\providecommand{\DIFaddendFL}{} %DIF PREAMBLE
\providecommand{\DIFdelbeginFL}{} %DIF PREAMBLE
\providecommand{\DIFdelendFL}{} %DIF PREAMBLE
%DIF END PREAMBLE EXTENSION ADDED BY LATEXDIFF
\usepackage[nolandscape]{mylists}

\author{Ryan Barnes}
\title{ODNP Workup Software}
\date{\today}
\begin{document}
\maketitle
\chapter{Introduction}
This guide outlines, in very rough detail, the how to of operating the ODNP workup software with python. Right now this is fairly unorganized but I would appreciate your help in making this work towards what you need it for. 

\chapter{Guide to Running the Workup Code}
Put instructions here for running the workup code. Initially go through the answers for the query statements, especially how to enter the dnp and t1 exp numbers. Then go through exceptions and how the user should use the command line to remedy the situation.

\chapter{Database Information}
I'm using mongodb to database all the worked up data according to a hierarchical scheme. This will allow a user to search the database for entries based on certain parameters. 

Currently the search headings include:
\begin{enumerate}
    \item \o{operator} - this is the experimenter's name e.g. Ryan Barnes.
    \item \o{date} - this is the date of the experiment
    \item \o{setType} - this is the data set type e.g. (kSigma, T1Series, enhancementSeries, T10)
    \item \o{macroMolecule} - this is the macromolecule in the sample e.g. protein, liposome. \o[If there is no macromolecule this is set to none.]{} 
    \item \o{bindingPartner} - this is the other macromolecule in sample that is not spin labeled. \o[If there is no other macromolecule in the sample this is set to none.]{} 
    \item \o{concentration} - this is the concentration of the \o{macroMolecule}, this assumes the spin label is the same concentration.
    \item \o{temperature(K)} - this is the temperature of the experiment set in Kelvin.
    \item \o{solvent} - this is the solvent for the experiment.
    \item \o{osmolyte} - This might be nice for Jinsuk for now this isn't used.
    \item \o{aggregationAgent} - this is more a place holder for experiments with the tau protein.
\end{enumerate}


\chapter{Development:}
\section{New Implementations:}
\begin{enumerate}
    \item $\Box$ You should add a check when a user changes key value. This check should run through the existing keys values to make sure that the value has already been entered, if it is a new key value the program should ask if this new key value is correctly entered, the idea is to catch mistakes in spelling. \o[Another way would be to let the user choose from a list of options instead of having to type in the key value.]{}
    \item $\checkmark$ Open a file dialog so the user can pick their experiment directory and the experiment file to work up. \o[This is done in concept, the minimum running example is in 'fileDialog.py'. You just need to implement and wrap into 'returnIntegrals.py']{}
    \item $\Box$ Make it so the notebook shows the operators name, not mine. \o[I don't think this is going to work like it should. Something weird with the latex package calls. Right now it just says Han Lab Notebook.]{}
    \item $\Box$ system calls that handle both windows and mac both in naming the file type and in compiling the pdf.
    \item $\Box$ Make a way to reload the freshly produced pdf. Mac will work with preview (you just need to set to use preview by default). Windows should work with Sumatra pdf --> This seems to work for mac nicely, need to add windows features now.
    \item $\Box$ Axis labels in the last plots generated by the code. This is important for kSigma because you want to give the appropriate units!
    \item $\Box$ You should make it so that the database dictionary keys are pulled from the live database and fill in the key values from the default file in the directory. Or keep a copy of the directory keys and if the current keys are different than the directory keys just update the directory keys. \o[It looks like you will have to home roll a function to print all keys.]{}
    \item $\checkmark$ When entering database values you should list all possible choices for the given key. Use '.distinct('keyValue')'. \o[You should make it so the database values that are returned are operator specific.]{}
    \item $\Box$ Add handlers for mistyped answers so program does not crash. Just say do not understand and re loop through the question in a while loop. \o[This still needs done for the error handling functions.]{}
    \item $\checkmark$ Change the experimental parameters queries to dynamic type like the database queries \o[This should be thoroughly debugged, it's not right now which might be a problem.]{This actually seems to work nicely now.}
    \item $\Box$ Newton's method to find starting guess for T1 experiments. For now the t1StartingGuess works 
    \item $\checkmark$ Make a way to force an experiment time for diving up the powers files \o[I pull the experiment time from the Bruker output and set this as the minimum experiment time. If the experiment is successful the workup software should work fine dividing up the powers.]{}
    \item $\checkmark$ You should calculate the experiment time from the Bruker timing function and hand this with an associated arror to the 'returnSplitPowers' function. \o[Right now this pulls the experiment time and uses as a lower bound for the split powers function.]{}
\end{enumerate}
\section{Bugs:}
\begin{enumerate}
    \item $\Box$ When the workup experiment crashes the database parameters dictionary is not saved. This is because it's pulling those parameters from the online dictionary and not from the file. \o[You should make it so you hand the returnDatabase function the name of the experiment. If the experiment name exists just hand back that database entry.]
    \item $\Box$ in the exception loop of the power failure there is no wrong answer handling.
    \item $\Box$ You should add a method to force a time for the power series.
    \item $\Box$ 'returnExpTimes' fails in windows when the experiment files aren't set right.
    \item $\checkmark$ 131115_tempcont_dnp_9_58GHz_jss gave power series error, it looks like the time steps are really off. \o[This is actually just a broken experiment.]{No need to adjust code to handle this. I did add functionality to the powers dividing function to drop the first number of time values with 'timeDropStart' and I also added a maximum experiment time as 'expTimeMax'. It might be worth adding debugging functionality to plot the time series and also show the time values of the spikes, this way the experimenter can pick out the values nicely.}
    \item $\Box$ 140509_200uM_OHT_ODNP gave an error because a glitch occuring early. You should add a way to throw away values that occur before a certain time.
    \item $\Box$ 140728_CheY_CtermP6C_400uM_ODNP_Repeat2 this throws an error because it can't line up the enhancement series and the powers file
    \item $\Box$ When code crashes due to not lining up the powers file with enhancemnet or $T_1$ series. The powers file is not returned in a csv.
    \item $\Box$ You need to add wrong answer handling to the power series exception handlers.
\end{enumerate}
\section{Methods to Implement:}
\begin{enumerate}
    \item $\Box$ Wrap method of saving the nddata sets to mongo database
    \item $\Box$ Method for print statement headers
\end{enumerate}
\section{Users Guide:}
\begin{enumerate}
    \item $\Box$ Windows installation of packages and use.
    \item $\Box$ Max installation of packages and use.
    \item $\Box$ Introduction to the software.
    \item $\Box$ Walk through of using the program.
    \item $\Box$ Common errors, reasons and how to handle them from the command line.
    \item $\Box$ Explination of functions used \o[This is a backend thing and is not necessary for now.]{}
\end{enumerate}
\chapter{Installation}
\section{Windows Install}
This details how to install the necessary software on windows. This will also go through how to get the program setup and running.
\subsection{Necessary Packages}
\begin{enumerate}
    \item Download \& install 'python xy' \o[Make sure to do the full install instead of the defaul custom install.]{}
        \begin{enumerate}
            \item Install 'pymongo' do this by typing 'easy_install pymongo' in the commandline. Note that if the command line yells at you that easy_install is not a recognized command you need to edit the path variable to inclue the python directory in the program files x86 directory. Google is yo friend.
        \end{enumerate}
    \item Download \& install 'git'
    \item Ask Ryan to share the 'workupSoftware' repository with you and follow the instructions for how to clone that repository to a directory of your choice.
    \item Install vim74 \o[This is for me (Ryan) for when I go play on your computer. If you install this you get major brownie points :)]
    \item Install notepad \o[This is for you and will make your life easier for when you need to edit the programs, which hopefully you wont have to do.]{Or you can just use the native spyder editor which comes with python xy}
    \item Install latex via texlive for windows. Just run the simple install. \o[This takes a long ass time so start this early.]{}
\end{enumerate}
\section{Getting Started}
\begin{enumerate}
    \item You need to edit the header parameter, for now this is in the code under header. This should point from 'C:/' to the folder that holds your data sets. \o[Mine looks like 'C:\\Users\\megatron\\exp_data\\ryan_cnsi\\nmr\\'.] I would open the file 'returnIntegrals.py' in a text editor like notepad and set 'header = C:\\Users\\megatron\\exp_data\\ryan_cnsi\\nmr\\' only with your information in place.
    \item 
\end{enumerate}
\chapter{EPR Integral Workup}
Is this necessary for the ODNP? It might be nice to do this as standard this way you can tell if the double integrals are the same. \o[This is a future idea.]

\end{document}
