\documentclass[10pt]{book}
\newif\ifshowall
\showalltrue
%\showallfalse
%\documentclass[twocolumn,nofootinbib]{revtex4}
\usepackage{cmap}
\usepackage[T1]{fontenc}
\usepackage{mynotebook}
%\usepackage[english,greek]{babel}
\synctex=1
%DIF PREAMBLE EXTENSION ADDED BY LATEXDIFF
%DIF UNDERLINE PREAMBLE %DIF PREAMBLE
\RequirePackage[normalem]{ulem} %DIF PREAMBLE
\RequirePackage{color}\definecolor{RED}{rgb}{1,0,0}\definecolor{BLUE}{rgb}{0,0,1} %DIF PREAMBLE
%\providecommand{\DIFadd}[1]{\begingroup\protect\color{blue}\uwave{#1}\endgroup} %DIF PREAMBLE
\providecommand{\DIFadd}[1]{#1} %DIF PREAMBLE
%\providecommand{\DIFdel}[1]{\begingroup\protect\color{red}\sout{#1}\endgroup}                      %DIF PREAMBLE
\providecommand{\DIFdel}[1]{#1}                      %DIF PREAMBLE
%DIF SAFE PREAMBLE %DIF PREAMBLE
\providecommand{\DIFaddbegin}{\color{blue}} %DIF PREAMBLE
\providecommand{\DIFaddend}{\color{black}} %DIF PREAMBLE
\providecommand{\DIFdelbegin}{} %DIF PREAMBLE
\providecommand{\DIFdelend}{} %DIF PREAMBLE
%DIF FLOATSAFE PREAMBLE %DIF PREAMBLE
\providecommand{\DIFaddFL}[1]{\DIFadd{#1}} %DIF PREAMBLE
\providecommand{\DIFdelFL}[1]{\DIFdel{#1}} %DIF PREAMBLE
\providecommand{\DIFaddbeginFL}{} %DIF PREAMBLE
\providecommand{\DIFaddendFL}{} %DIF PREAMBLE
\providecommand{\DIFdelbeginFL}{} %DIF PREAMBLE
\providecommand{\DIFdelendFL}{} %DIF PREAMBLE
%DIF END PREAMBLE EXTENSION ADDED BY LATEXDIFF
\usepackage[nolandscape]{mylists}

\author{Ryan Barnes}
\title{ODNP Workup Software}
\date{\today}
\begin{document}
\maketitle
\section{Introduction}
This guide outlines, in very rough detail, the how to of operating the ODNP workup software with python. Right now this is fairly unorganized but I would appreciate your help in making this work towards what you need it for. 

\section{Random Collection of Thoughts and Notes for now}
Really I'm sorry to do it like this for now but I'm not sure of an organization so for the time being this will have to do.

After you run the code $'returnIntegrals.py'$ and $'plots.pdf'$ opens scroll through the $T_1$ series plots and make sure the fits look good. If you see that one or many fits did not converge change the parameter \o[$'t1StartingGuess'$]{} in $'returnIntegrals.py'$, this is the starting guess for the $T_1$ value (in seconds) of the fit as long as it's within a second of the actual value it should be good.

\section{EPR Integral Workup}
Is this necessary for the ODNP? It might be nice to do this as standard this way you can tell if the double integrals are the same.

\section{Database Information}
I'm using mongodb to database all the worked up data according to a hierarchical scheme. This will allow a user to search the database for entries based on certain parameters. 

Currently the search headings include:
\begin{enumerate}
    \item \o{operator} - this is the experimenter's name e.g. Ryan Barnes.
    \item \o{date} - this is the date of the experiment
    \item \o{setType} - this is the data set type e.g. (kSigma, T1Series, enhancementSeries, T10)
    \item \o{macroMolecule} - this is the macromolecule in the sample e.g. protein, liposome. \o[If there is no macromolecule this is set to none.]{} 
    \item \o{bindingPartner} - this is the other macromolecule in sample that is not spin labeled. \o[If there is no other macromolecule in the sample this is set to none.]{} 
    \item \o{concentration} - this is the concentration of the \o{macroMolecule}, this assumes the spin label is the same concentration.
    \item \o{temperature(K)} - this is the temperature of the experiment set in Kelvin.
    \item \o{solvent} - this is the solvent for the experiment.
    \item \o{osmolyte} - This might be nice for Jinsuk for now this isn't used.
    \item \o{aggregationAgent} - this is more a place holder for experiments with the tau protein.
\end{enumerate}


\end{document}
